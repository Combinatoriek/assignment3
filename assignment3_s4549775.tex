\documentclass[12pt]{article}

\usepackage{amsmath}
\usepackage{amssymb}
\usepackage{enumerate}
\usepackage[margin=5em]{geometry}

\setlength\parindent{0em}
\setlength\parskip{1em}

\title{Assignment 3}

\author{
	Hendrik Werner s4549775
}

\begin{document}
\maketitle

\section*{11}
All positive integers below 1000000 have 6 digits: $[000001, 999999]$. Exactly one digit needs to be 9, so we have 6 possible positions to place it.

After that we are left with 5 places, that need to add up to 4, so the sum of digits is 13. There are $\binom{5 - 1 + 4}{4}$ possibilities.

In total there are $6 * 70 = 420$ numbers that fulfill the requirements.

\section*{12}
We put distinguishable objects (playing cards) into labeled boxes (players hands). Therefore there are $\dfrac{52!}{7!^5 * (52 - 5 * 7)!} =  \dfrac{52!}{7!^5 * 17!}$ possibilities to do that.

\section*{13}
\begin{enumerate}[a]
	\item %a
	The number of books needs to add up to 12, but all books are the same. There are $\binom{4 - 1 + 12}{4 - 1} = 455$ possibilities to place the books.

	\item %b
\end{enumerate}

\section*{14}
\begin{enumerate}[a]
	\item %a
	If boxes and balls are labeled than the order of balls does matter. There are $P(5, 3) = 60$ possibilities to put 3 out of 5 labeled balls into 3 labeled boxes. For the remaining two balls, we have 3 choices each.

	The total number of possible distributions is $60 * 3^2 = 540$.

	\item %b
	Since the boxes are unlabeled the order we put the first 3 balls into them does not matter, we just need to choose which of the 5 balls we use for it: $\binom{5}{3} = 10$ possibilities.

	Now that the boxes are not empty anymore, we can distinguish them by the labeled balls that are already in them. The remaining two balls can be distributed in 3 ways each.

	Total number of possible distributions: $10 * 3^2 = 90$.
	\item %c
	\item %d
\end{enumerate}

\section*{15}

\section*{16}

\section*{17}
\begin{enumerate}[a]
	\item %a
	\item %b
	\item %c
\end{enumerate}

\end{document}
