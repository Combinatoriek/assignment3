\documentclass[12pt]{article}

\usepackage{amsmath}
\usepackage{amssymb}
\usepackage{enumerate}
\usepackage[margin=5em]{geometry}

\setlength\parindent{0em}
\setlength\parskip{1em}

\title{Assignment 3}

\author{
	Hendrik Werner s4549775
}

\begin{document}
\maketitle

\section*{11}
All positive integers below 1000000 have 6 digits: $[000001, 999999]$. Exactly one digit needs to be 9, so we have 6 possible positions to place it.

After that we are left with 5 places, that need to add up to 4, so the sum of digits is 13. There are $\binom{5 - 1 + 4}{4}$ possibilities.

In total there are $6 * 70 = 420$ numbers that fulfill the requirements.

\section*{12}
We put distinguishable objects (playing cards) into labeled boxes (players hands). Therefore there are $\dfrac{52!}{7!^5 * (52 - 5 * 7)!} =  \dfrac{52!}{7!^5 * 17!}$ possibilities to do that.

\section*{13}
\begin{enumerate}[a]
	\item %a
	The number of books needs to add up to 12, but all books are the same. There are $\binom{4 - 1 + 12}{4 - 1} = 455$ possibilities to place the books.

	\item %b
\end{enumerate}

\section*{14}
\begin{enumerate}[a]
	\item %a
	The possible distributions of balls can be either one box with 3 balls and two boxes with 1 ball, or one box with 1 ball and two boxes with 2 balls.

	In the first case we need to choose the bigger box $\binom{3}{1} = 3$, then choose which 3 balls to put into it $\binom{5}{3} = 10$, then choose which balls go into which remaining box $\binom{2}{1} \binom{1}{1} = 2$. There are $3 * 10 * 2 = 60$ ways to do the first case.

	In the second case we need to choose the smaller box $\binom{3}{1} = 3$, then choose which ball goes into it $\binom{5}{1} = 5$, and finally fill the remaining boxes $\binom{4}{2} \binom{2}{2} = 6 * 1 = 6$. The second case can be done in $3 * 5 * 6 = 90$ ways.

	In total there are $60 + 90 = 150$ ways to divide the balls.

	\item %b
	Since the boxes are unlabeled the order we put the first 3 balls into them does not matter, we just need to choose which of the 5 balls we use for it: $\binom{5}{3} = 10$ possibilities.

	Now that the boxes are not empty anymore, we can distinguish them by the labeled balls that are already in them. The remaining two balls can be distributed in 3 ways each.

	Total number of possible distributions: $10 * 3^2 = 90$.
	\item %c
	First we put one ball into each box. Since the balls are unlabeled there is only one way to do this. After that we have two balls left which we can put into any of the boxes. There are $\binom{3 - 1 + 2}{2} = 6$ ways to do the second step.

	In total we can divide the balls in $1 * 6 = 6$ ways.
	\item %d
\end{enumerate}

\section*{15}

\section*{16}
We can look at the 7 different bills as labeled objects, and the purses as 3 unlabeled boxes. At most one purse may be empty. There are 2 scenarios:

\begin{enumerate}
	\item
	Divide the notes between 2 purses. There are $S(7, 2) = 63$ ways to do this.
	\item
	Divide the notes between 3 purses. There are $S(7, 3) = 301$ ways to to this.
\end{enumerate}

In total we can divide the bills in $63 * 301 = 18963$ ways.

\section*{17}
\begin{enumerate}[a]
	\item %a
	The student has 9 fruits. He first chooses on which of the 9 days to each the apples. There are $\binom{9}{4} = 126$ possibilities, because all apples are the same. After that he chooses on which of the remaining days to eat the bananas. There are $\binom{5}{3} = 10$ possibilities. Now there are two pears left, which the student eats on the other 2 days.

	In total he can divide the fruit in $126 * 10 = 1260$ ways.
	\item %b
	A client can choose any of the 16 flavors for each of the 3 scoops, so there are $16^3 = 4096$ different possibilities. Next he can choose 2 out of the 7 sauces, but he cannot reuse them. There are $7 * 6 = \binom{7}{2} = 42$ combinations. Choosing 2 out of the 5 toppings without repetition follows the same principle as the sauce. There are $\binom{5}{2} = 10$ possibilities.

	All the steps can be combined in any way, so there are $4096 * 42 * 10 = 1720320$ different medium sundaes, which the clients can make.
	\item %c
	$A = \{0, 1, 2, 3, 4, 5\} = 6, B = \{a, b, c, d, e, f\}$

	Every element from the domain $A$ needs to be mapped onto one element from the codomain $B$, but $|A| > |B|$, so according to the pigeonhole principle, there needs to be element on the codomain that is mapped to by 2 elements from the domain. There cannot be more than one such element, because the function needs to be surjective.

	There are $C(6, 2) = 15$ partitions with 4 sets of length 1 and 1 set of length 2. Now we need to choose an order for the codomain. $|B| = 5$, so there are $P(5, 5) = 5! = 120$ permutations.

	In total there are $15 * 120 = 1800$ surjective functions of type $A \rightarrow B$.
\end{enumerate}

\end{document}
