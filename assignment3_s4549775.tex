\documentclass[12pt]{article}

\usepackage{amsmath}
\usepackage{amssymb}
\usepackage{enumerate}
\usepackage[margin=5em]{geometry}

\setlength\parindent{0em}
\setlength\parskip{1em}

\title{Assignment 3}

\author{
	Hendrik Werner s4549775
}

\begin{document}
\maketitle

\section*{11}
All positive integers below 1000000 have 6 digits: $[000001, 999999]$. Exactly one digit needs to be 9, so we have 6 possible positions to place it.

After that we are left with 5 places, that need to add up to 4, so the sum of digits is 13. There are $\binom{5 - 1 + 4}{4}$ possibilities.

In total there are $6 * 70 = 420$ numbers that fulfill the requirements.

\section*{12}
We put distinguishable objects (playing cards) into labeled boxes (players hands). Therefore there are $\dfrac{52!}{7!^5 * (52 - 5 * 7)!} =  \dfrac{52!}{7!^5 * 17!}$ possibilities to do that.

\section*{13}
\begin{enumerate}[a]
	\item %a
	The number of books needs to add up to 12, but all books are the same. There are $\binom{4 - 1 + 12}{4 - 1} = 455$ possibilities to place the books.

	\item %b
\end{enumerate}

\section*{14}
\begin{enumerate}[a]
	\item %a
	If boxes and balls are labeled than the order of balls does matter. There are $P(5, 3) = 60$ possibilities to put 3 out of 5 labeled balls into 3 labeled boxes. For the remaining two balls, we have 3 choices each.

	The total number of possible distributions is $60 * 3^2 = 540$.

	\item %b
	Since the boxes are unlabeled the order we put the first 3 balls into them does not matter, we just need to choose which of the 5 balls we use for it: $\binom{5}{3} = 10$ possibilities.

	Now that the boxes are not empty anymore, we can distinguish them by the labeled balls that are already in them. The remaining two balls can be distributed in 3 ways each.

	Total number of possible distributions: $10 * 3^2 = 90$.
	\item %c
	\item %d
\end{enumerate}

\section*{15}

\section*{16}
We can look at the 7 different bills as labeled objects, and the purses as 3 unlabeled boxes. At most one purse may be empty. There are 2 scenarios:

\begin{enumerate}
	\item
	Divide the notes between 2 purses. There are $?$ ways to do this.
	\item
	Divide the notes between 3 purses. There are $?$ ways to to this.
\end{enumerate}

\section*{17}
\begin{enumerate}[a]
	\item %a
	The student has 9 fruits. He first chooses on which of the 9 days to each the apples. There are $\binom{9}{4} = 126$ possibilities, because all apples are the same. After that he chooses on which of the remaining days to eat the bananas. There are $\binom{5}{3} = 10$ possibilities. Now there are two pears left, which the student eats on the other 2 days.

	In total he can divide the fruit in $126 * 10 = 1260$ ways.
	\item %b
	A client can choose any of the 16 flavors for each of the 3 scoops, so there are $16^3 = 4096$ different possibilities. Next he can choose 2 out of the 7 sauces, but he cannot reuse them. There are $7 * 6 = \binom{7}{2} = 42$ combinations. Choosing 2 out of the 5 toppings without repetition follows the same principle as the sauce. There are $\binom{5}{2} = 10$ possibilities.

	All the steps can be combined in any way, so there are $4096 * 42 * 10 = 1720320$ different medium sundaes, which the clients can make.
	\item %c
	$A = \{0, 1, 2, 3, 4, 5\} = 6, B = \{a, b, c, d, e, f\}$

	First we map one element from the domain A to each of the elements in the codomain B, so that the function is surjective. There are $P(|A|, |B|) =  P(6,5) = 720$ ways to do this. Functions map each element of the domain to something, so we need to choose one of 5 elements in the codomain to be mapped to by the last element in the domain.

	In total there are $720 * 5 = 3600$ surjective functions of type $A \rightarrow B$.
\end{enumerate}

\end{document}
